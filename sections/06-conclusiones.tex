\chapter{\textcolor{azulescom}{Conclusiones}}

El desarrollo de una aplicación web para la predicción de precios de
apartamentos, como SkyPrice, ha demostrado ser una herramienta valiosa y
eficiente para los usuarios interesados en el mercado inmobiliario. A través
de este proyecto, se ha logrado integrar diferentes tecnologías y metodologías
avanzadas, destacando el uso de algoritmos de inteligencia artificial y aprendizaje
automático para proporcionar estimaciones precisas y confiables.

La plataforma SkyPrice fue diseñada con un enfoque en la accesibilidad y la
usabilidad, permitiendo a los usuarios acceder a las predicciones desde cualquier
dispositivo con conexión a internet y un navegador web. Además, se implementó
un diseño intuitivo y moderno, facilitando la navegación y la interacción del
usuario con la aplicación. Las instrucciones claras en cada paso del proceso
garantizan una experiencia amigable y satisfactoria.

El modelo de predicción utilizado en SkyPrice se entrenó con un conjunto de
datos representativo del mercado inmobiliario, lo que permitió obtener una
precisión significativa en las estimaciones de precios. Los resultados del modelo
mostraron una alta correlación entre los precios predichos y los precios reales
de los apartamentos, validando la efectividad del enfoque adoptado.

Además, se analizaron los datos de interacción de los usuarios dentro de la
plataforma, proporcionando valiosa información sobre las secciones más relevantes
y el impacto de la aplicación en línea. Estos datos no solo ayudaron a mejorar
la experiencia del usuario, sino que también proporcionaron insights sobre las
tendencias y preferencias del mercado inmobiliario.

En conclusión, SkyPrice no solo facilita la toma de decisiones informadas para
compradores y vendedores de bienes raíces, sino que también demuestra el
potencial de las tecnologías emergentes para transformar y optimizar procesos
en diversas industrias. El éxito de este proyecto subraya la importancia de
seguir explorando y aplicando innovaciones tecnológicas para resolver problemas
reales y mejorar la calidad de vida de las personas.

